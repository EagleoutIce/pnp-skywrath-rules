\documentclass{pnp-skywrath-documentclass/pnp-skywrath}

\usepackage{tikzsymbols,imakeidx,wrapfig,fontawesome}
\usepackage[english, main=ngerman]{babel}

\input{shortcuts.src}

% index setup
\makeindex[title=Schlagworte,options=-s configs/index,columns=2,columnsep=0.75cm]\renewcommand{\indexname}{Schlagworte}

\title{Das Skywrath-Regelwerk}
\subtitle{Für das P\&P \say{Das Leid der Anderen}}
\author{Florian Sihler}

\toctitle{Content}

\definecolor{cprimary}{RGB}{195,33,72}


\begin{document}
    \maketitle
    \TableOfContents

    \begin{abstract}[Achtung:]
Aus einer Mischung aus Zeitgründen und Nachfrage ist diese weitere Zwischenversion entstanden in der immer noch ordentlich was fehlt, die aber hinsichtlich der Charaktererstellung weitere wichtige Punkte vermittelt. Bitte nochmal ganz lesen. Es hat sich bisschen was auch in dem bereits bestehenden geändert!\\
\hbox{}\hfill{}{\footnotesize Fassung vom: \today}
    \end{abstract}

    \section{Ein Vorwort}
    \label{sec:vorwort}Erstmal ja, das hier ist schon ein langer, seeehr langer Text, den ich mit all den Wortwitzen und Versuchen ihn aufzulockern vermutlich nur noch schlimmer und grausiger zu lesen gemacht habe. Aber: Es ist nun mal klausurrelevant, also müssen wir da alle irgendwie durch\ldots 

Diese Vorwort widmet sich hier nur der Illustration beziehungsweise eurer Mitarbeit und einem großen und gütigem Angebot welches ich euch in diesem Hinblick unterbreite:
Überall in diesem Regelwerk habe ich fehlende Grafiken und unfertige Texturen versteckt, die ihr gerne mit euren Fertigkeiten in der Kunst befüllen und veredeln dürft. Wenn ihr also zum Beispiel die hier aufgeführten Zauber (alles unterkringelte ist hierbei nur kurz angemerkt in den meisten PDF-Ansichtsprogrammen anklickbar :D) mit Bildchen versehen wollt, so seid herzlich dazu angestiftet und bereits im vor raus mit Dank überschüttet.
Gleiches gilt allerdings auch für Objekte über das Regelwerk hinaus! Sofern ihr dies wollt (und ich werde mich hier einfach nach einer, für alle verpflichtenden, Mehrheitsentscheidung richten) werde ich die Items die ihr erhaltet -- mit den dazugehörigen Werten -- auf einen Zettel schreiben (bzw. auch das dürft natürlich ihr tun, ganz wie es euch beliebt) und euch die Möglichkeit geben, diese mit illustren Bildern ganz in eurem Stil zu versehen. Dies bedeutet nicht, dass ihr, sobald ihr ein Item in eure gierigen Griffel bekommt, sofort mit dem pinseln anfangen sollt, aber es soll euch gewährt sein in Zeiten der Langeweile oder auch zwischen den Abenteuern das Handwerk der Kunst auszuführen. Das ermöglicht nicht nur eine erweiterte Atmosphäre, sondern auch weniger komplettes abdriften und weniger Nebengespräche und eröffnet zugleich die Möglichkeit dass wir das Abenteuer am Ende verkaufen und damit Reich werden! -- Sorry, abgedriftet, Raphael ist ein schlechter Einfluss \Laughey.

Ich habe auch schon viel zu lange geredet, deswegen seid ihr hiermit in den folgenden Roman entlassen. Lehnt euch zurück, schnappt euch ne Tüte überteuertes Popcorn, Stifte raus und mitgeschrieben, auf-dass ein wundervolles Pen and Paper Abenteuer gedeiht\ldots


Mit der Übersetzung des Regelwerk in \LaTeXe{} gibt es eine Menge Verbesserungen. Darunter einige \link[sec:vorwort]{Querverweise} wie auch einen \link[sec:index]{Index} am Ende des Dokuments.

    \section{Die Charaktererstellung}
    %
\def\cimp#1{\index{Charakter!Attribute!#1}\imp{#1}}
%
\begin{quote}
    Wir begrüßen dich, Hans, Advokat der ersten Senf-Brigade.
\end{quote}
\subsection{Allgemeines}
\index{Charakter}\index{Charakter!Attribute}Die grundlegenden Eigenschaften eines Charakters ändern sich in diesem Regelwerk nicht wirklich. Überraschung!

Ihr braucht einen \cimp{Namen}, dessen Zusammensetzung irgendwie ins Setting passen sollte. Zudem erfragt der Charakterbogen das \cimp{Alter}, die \cimp{Größe}, das \cimp{Gewicht} und die \cimp{Hautfarbe} eures Charakters. Während die Stimmigkeit der ersten drei (im Bezug auf eure anderen Fertigkeiten) euch überlassen ist, so kann eure Hautfarbe im Bezug zum Setting, rassistisch bedingt, Konsequenzen haben.
Ich empfehle euch, keinen dunkelhäutigen Charakter zu spielen, da dies in der Öffentlichkeit zu Problemen führen kann, möchte euch aber diesbezüglich keinesfalls einschränken.

Weiter sind eure \cimp{Lebenspunkte} von Relevanz.
Ihr alle habt grundsätzlich \(40\) Lebenspunkte.
Diese lassen sich selbstverständlich erhöhen und sie erhöhen sich auch mit eurem \cimp{Level}, wie genau das allerdings im Detail abläuft erfahrt ihr im Kapitel \say{\link[sec:charakterbogen]{Charakterbogen}}.
Elementare Bestandteil eures Wesens ist eure \cimp{Profession}, die vereinfacht das Wesen eures Charakters repräsentiert.
Diese hilft mir bei Fragen zu verstehen was ihr mit eurem Charakter erreichen und welche Rolle ihr einnehmen wollt, ohne euch in die klassische Assassinen-Kampfmagier-Affenzähmer-Virtuosen Klasse einzuteilen. Die Profession ist allerdings auch so relevant, da ich euch darüber und über euer Charakterdesign doch in eine Schublade stecke und dementsprechend Boni verteile. Dies sollte euch helfen, trotz der wenigen Punkte die es zu Beginn gibt, euren Charakter zu verwirklichen. 

Aufgewertet wird euer Charakter, neben einem wunderschönen \cimp{Bild}, durch eure \cimp{Hintergrundgeschichte} (\link[sec:charakterbogen]{Charakterbogen}).
Diese ist in zwei Teile aufgebaut.
Einer der Beiden befindet sich auf eurem Charakterbogen und symbolisiert das von eurem Charakter, was ihr als dieser Preisgebt, bzw. Leute relativ schnell erfahren.
Den anderen, meist durchaus größeren Teil, dürft ihr aufschreiben wie es euch beliebt. 
-- Dieser sollte aber auch mir irgendwie übermittelt werden, damit ich während des Spiels darauf eingehen kann\footnote{Natürlich steht es euch frei den Teil dennoch zu verbreiten, das sollte ich dann nur wissen.}.
Es ist durchaus möglich, dass die Geschichte die ihr auf eurem Charakterbogen verfasst nicht mit dem Extra-Blatt übereinstimmt. Auf dem angegeben Extra-Blatt muss dann aber eure wahre Hintergrundgeschichte stehen, die ihr im Verlauf des PnP auch gerne preisgeben könnt. 
Aber bleibt realistisch, niemand setzt sich so einfach jemandem gegenüber und erzählt von seiner \say{harten Schule einer echten Waise} seitdem er damals aus der Wiege gefallen ist (hust Morgengrau hust \Tongey).
Bitte beachtet, dass es keine gesellschaftliche Stellung gibt, da ihr ja alle in der Gang gelandet seid und diese dementsprechend klar ist. Sollte die Stellung in zukünftigen Sitzungen von Relevanz sein, füge ich sie gegebenenfalls hinzu.




    \clearpage
    \section{Der Charakterbogen}
    \label{sec:charakterbogen}\begin{quote}
    Ach ich darf mein Leben gar nicht für die Punkte opfern?
\end{quote}
\subsection{Allgemeine Daten}

Hierbei sei erst einmal gesagt, dass das endgültige Layout des Charakterbogens noch nicht fertig ist (und ich gerne Designs/Zeichnungen von euch darin einbaue um das Ganze aufzuhübschen \Laughey) und ihr deswegen hier nur pragmatische Darstellungen der jeweiligen Sektion seht, die nur aus semantischer Sicht so übernommen werden.

\index{Punkte}Wichtig ist, dass es diesmal \index{Punkte!Spezifisch (\SP)}spezifische (\SP{}) und \index{Punkte!Global/Magisch (\GP)}globale (\GP{}) Punkte gibt. \SP{} sind an eine Fertigkeiten-Gruppe gebunden. \SP{} können \(4:1\) in \GP{} umgewandelt und damit auch in anderen Gruppen eingesetzt werden (\GP{} sind allerdings nicht mehr Wert, die einzige Ausnahme hiervon ist die Magie\ldots).% TODO: link Magie

\paragraph{Allgemeine Metadaten}
\label{par:charMeta}\begin{sccenter}
    \begin{tikzpicture}[lc/.style={below,scale=0.65,font=\itshape}]
        \draw[rounded corners] (0,0) rectangle ++(13,1.3);
        \draw (0.5,0.5) -- ++(5,0) node[lc,midway] {Name} node[above,midway] {\strut{Willey Jefferson}};
        \draw (6,0.5) -- ++(6.5,0);
        \foreach[count=\i] \a/\b in {12/Alter,1.75/Größe,70\,kg/Gewicht,Grün/Hautfarbe} {
            \node[lc] at(7+1.5*\i-1.5,0.5) {\b}; 
            \node[above] at(7+1.5*\i-1.5,0.5) {\strut\a}; 
        }
    \end{tikzpicture}
\end{sccenter}
Die Daten, die dort eingetragen werden sollen, sollten ohne große Erklärung funktionieren. Deswegen hopp hopp, und weiter geht die Reise\ldots

\paragraph{Die Lebensversicherung}
\begin{wrapfigure}[4]{r}{0.375\linewidth}
    \vspace*{-0.75\topsep}\centering\begin{tikzpicture}[lc/.style={below,scale=0.65,font=\itshape}]
        \draw[rounded corners] (0,0) rectangle ++(5.75,1.2);
        \draw (0.5,0.5) -- ++(2.25,0) node[lc,midway] {HP (2:1)} node[above,midway] {40/{\smaller40}};
        \draw (3,0.5) -- ++(2.25,0) node[lc,midway] {MP (--:--)} node[above,midway] {20/{\smaller20}};
    \end{tikzpicture}
\end{wrapfigure}
Erstmal zu den \imp{Lebenspunkten (HP)}: 
Ihr startet mit 40, was nach verhältnismäßig wenig klingt aber schon ganz ordentlich ist im Verhältnis zum allgemeinen Schadenssystem. % TODO: LinK
Die Kosten für Leben betragen \(2:1\) und können aus allen anderen Gruppen zusammengespart werden. Der Verkaufspreis ist invertiert, ein Lebenspunkt liefert also zwei Fertigkeitenpunkte (\SP).
Die Lebenspunkte dürfen durch das verkaufen nicht auf oder unter Null gesetzt werden. Weniger Lebenspunkte werden aber auch hart bestraft\ldots

Eure Lebenspunkte sind den folgenden Regeln unterworfen:
\begin{itemize}
    \item \label{char:createNew}Fällt die Anzahl der Lebenspunkte aus einem beliebigen Grund auf oder unter \(0\) so scheidet der Charakter umgehend aus. In diesem Fall ist ein neuer Charakter zu erstellen, dessen Level zwei unter dem des bisher niedrigsten Gruppenlevels ist\footnote{Das Level kann über diese Mechanik nicht unter 1 gesetzt werden.}.
    \item Verursacht ein Angriff mindestens so viel Schaden wie die Hälfte der aktuellen Lebenspunkte, so ist eine Probe zu werfen (hier erscheint eine Referenz, sobald es diese Probe gibt). % TODO: link
    \item Fällt die Anzahl der Lebenspunkte unter 5, so gilt der Charakter als \imp{schwer verwundet} und kann eigenhändig keine Aktionen wie laufen, oder gar kämpfen durchführen.
    \item Jegliche Form von Heilung (egal ob magisch oder physisch) lässt sich für jede Wunde nur einmal pro Tag\footnote{Solche Begrenzungen setzen wir lose auf \say{alle \(24\)} Stunden, nicht das einer um Mitternacht direkt zwei Bandagen anbringen möchte.} anwenden.
    \item \index{Schlaf}Eine vom Spielmeister als \say{angenehm} eingestufte Nacht (in der geschlafen wurde) stellt \(\dice{1}{6} + 1\) Leben wieder her, sofern am Tag mindestens einmal am Tag gegessen wurde. Es steht dem Spielmeister frei, Schaden für das Unterlassen einer Nahrungsaufnahme zu verteilen.
\end{itemize}

Die \imp{Manapunkte (MP)} sind auf 20 beschränkt. Das Limit kann weder durch \SP/\GP{}, noch durch einen Levelaufstieg verändert werden.
Es existieren Gegenstände und Tier die selbst Mana innehaben und so zum wiederauffüllen verwendet werden können. Der Pool ist für den Anwender aber stets auf 20 eingegrenzt\footnote{So ist auch Magie in die beliebig viele Manapunkte investiert werden können auf maximal 20 Manapunkte beschränkt. Weitere \say{Manapunkte} können aber durch die gegebenen Regeln aufgebracht werden}. Für die Manapunkte gelten die folgenden Regeln:
\begin{itemize}
    \item Fällt das Mana, aus irgendeinem Grund, auf 0, so schließt dies nicht aus weiter Magie einzusetzen. Für je einen Lebenspunkte kann je ein weiterer Manapunkt investiert werden. So ist es auch möglich weiter zu zaubern, wenn der Mana-Pool aufgebraucht ist. Ein Magieanwender kann durch diese Mechanik zu Tode kommen.
    \item Schlaf generiert \emph{keine} Manapunkte.
    Magier können anstelle zu Schlafen auch \index{Meditation}\imp{Meditieren} und erhalten \(\dice{1}{4} + 1\) Manapunkte für eine \say{angnehme Nacht} zurück.
\end{itemize}

\paragraph{Level}
\begin{wrapfigure}[4]{r}{0.375\linewidth}
    \vspace*{-0.75\topsep}\centering\begin{tikzpicture}[lc/.style={below,scale=0.65,font=\itshape}]
        \draw[rounded corners] (0,0) rectangle ++(5.75,1.2);
        \draw (0.5,0.5) -- ++(2.25,0) node[lc,midway] {Level} node[above,midway] {4};
        \draw (3,0.5) -- ++(2.25,0) node[lc,midway] {XP \(\mathrm{LVL} \cdot 100\)} node[above,midway] {56/{\smaller300}};
    \end{tikzpicture}
\end{wrapfigure}
Ein Charakter startet auf \imp{Level (LVL)} 1, sofern er nicht durch die \say{\link[char:createNew]{Neuerstellungsmechanik}} auf einem höheren Level beginnt.
Die \imp{Erfahrungspunkte (XP)} beginnen in jedem Falle bei \(0\).
Um ein Level aufzusteigen werden jeweils \(\mathrm{LVL} \cdot 100\) Erfahrungspunkte benötigt.
Um von Level 1 auf Level 2 aufzusteigen werden also 100, um von Level 2 auf Level 3 aufzusteigen werden 200 Erfahrungspunkte benötigt, und so weiter\ldots

Für den \imp{Levelaufstieg} von Level 1 auf Level 2 werden \SP[6] pro Kategorie und \GP[3] vergeben.
Jeder Aufstieg darüber wird mit \SP[6] in jeder Kategorie und \GP[4] vergütet.

\subsection{Ausrüstung}

Zusätzlich zu eurem super duper gezeichneten \imp{Bild} des Charakters\footnote{An das Bild ist eine hohe Erwartungshaltung gerichtet ist, schließlich soll das Verlieren eines charakters ja auch richtig weh tun \Winkey.} geht es nun einmal um die \imp{Ausrüstung} welches sein Antlitz teils überdeckt.
Auch wenn es hier etliche Segmente gibt, braucht ihr nicht für jede ein eigenes Rüstungsteil. Ihr könnt euch die \say{Deko}-Kleidung aussuchen. Diese Kleidung hat aber
keine Werte (ist also wirklich nur kosmetischer Natur) und hilft auch in Debatten wie \say{Aber schau mal auf dem Bild trägt der eindeutig ein Jetpack} nicht weiter.
Jeder der folgenden Slots kann mit einem Element/Kleidungsstück belegt werden welches
Rüstungspunkte und sogar weitere Vorteile mit sich bringen mag. Solche Gegenstände gilt
es grundlegend zu erwerben. % TODO: link shopping
\begin{description}
    \item[Gesicht:] Platz für Masken, Sonnenbrillen, \ldots{} Dieser Slot kann (so dem Willen des Spielleiters) auch diverse Gegenstände fassen.
    \item[Kopfbedeckung:] Helme, Hüte, Perücke, \ldots{} Auch dieser Slot kann diverse Gegenstände fassen.
    \item[Mantel:] Jacken, Mäntel, Capes, \ldots{} Wird hier beispielsweise ein Wintermantel für die Kälte getragen kann dies am Tag für Nachteile sorgen. Solche Gegenstände können dann natürlich im Inventar verstaut werden.
    \item[Oberkörper:] Hemd, Bluse, Rüstung, \ldots{}
    \item[Hüfte:] Köcher, Wurfmesser, Munition \ldots{} Es kann auch als \say{Quick-Recharge-Slot} betrachtet werden. Im Kampf legt schließlich keiner den Rucksack auf dem Boden und sucht nach Munition.
    \item[Handschuhe:] Krallen, Handschuhe, \ldots{}
    \item[Beine:] Lange- und Kurze-Hose, Rock, \ldots{}
    \item[Füße:] Schuhwerk. Hier haben auch kosmetische Schuhe eine Wirkung. Keine Sohle und heißer Sand ist jetzt nicht die erfreulichste Kombination.
\end{description}

Doch wohin gehören nun meine Waffen und die ganzen Schildkrötenkuscheltiere? Für diese gibt es separate Slots.
Der Einfachheit halber einigen wir uns darauf, dass \emph{zwei Waffen} und \emph{zwei weitere Gegenstände} (die auch Waffen sein könnten) so getragen werden können, dass sie im Kampf schnell zur Verfügung stehen können\footnote{Sicherlich lässt sich über die Zahl der Gegenstände diskutieren, die hier aufgeführten Zahlen dienen mehr als Richtwerte für Gegenstände die wir direkt am Körper tragen.}.

Wir nehmen übrigens stets an, dass ihr nicht mit offenen Waffen herumlauft.
Die Konfiguration die ihr wählt kann aber durchaus dafür sorgen, dass ihr Waffen sichtbar tragt.
Die Waffen nicht offen zu tragen heißt aber nicht gleichsam, dass ihr sie versteckt geschweigedenn gut versteckt.

\subsection{Charakterwerte}

\paragraph{Angriffswerte}
\begin{wrapfigure}[4]{r}{0.275\linewidth}
    \vspace*{-0.75\topsep}\centering\begin{tikzpicture}[lc/.style={below,scale=0.65,font=\itshape}]
        \draw[rounded corners] (0,0) rectangle ++(4.25,1.2);
        \draw (0.5,0.5) -- ++(1.5,0) node[lc,midway] {ATK} node[above,midway] {3};
        \draw (2.25,0.5) -- ++(1.5,0) node[lc,midway] {INI} node[above,midway] {4};
    \end{tikzpicture}
\end{wrapfigure}

\def\awimp#1{\index{Charakter!Angriffswerte!#1}\imp{#1}}
\index{Charakter!Angriffswerte}Für den Angriff gibt es zwei grundlegende Werte, die \awimp{Attacke (ATK)} wie auch \awimp{Initiative INI}.
Für letztere sind niedere Werte besser, da ein Zug bei jedem vielfachen des hochzählenden Initiativwertes ausgeführt werden darf. 
Beide Werte werden von mir festgelegt und werden auf Basis der anderen
Fertigkeiten im Folgenden berechnet\footnote{Die genaue Formel steht noch aus.}.

\paragraph{Hauptwerte}
\begin{sccenter}
    \begin{tikzpicture}[lc/.style={below,scale=0.65,font=\itshape}]
        \draw[rounded corners] (0,0) rectangle ++(10,1.3);
        \foreach[count=\i] \a/\b in {9/STR,7/GES,7/KON,8/INT,10/WEI,7/CHR} {
            \node[lc] at(1.25+1.5*\i-1.5,0.5) {\b}; 
            \node[above] at(1.25+1.5*\i-1.5,0.5) {\strut\a}; 
        }
        \draw (0.5,0.5) -- ++(9,0);
    \end{tikzpicture}
\end{sccenter}
Es gibt sechs große Werte die jeweils auf eine \dice{}{20} Probe abzielen. 

\def\hwimp#1{\index{Charakter!Hauptwerte!#1}\imp{#1}}
\index{Charakter!Hauptwerte}Wir kennen die folgenden sechs Werte: \hwimp{Stärke (STR)}, \hwimp{Geschicklichkeit (GES)}, \hwimp{Konsitution (KON)}, \hwimp{Intuition (INT)}, \hwimp{Weisheit (WEI)}, \hwimp{Charisma (CHR)}.
Alle diese Werte starten mit einer 7, zu Beginn habt ihr 6 Punkte, die
ihr \(1:1\) auf diese verteilen dürft.
Anschließend können \GP{} eingesetzt werden um die Punkte weiter zu steigern.
% TODO: skill up

\paragraph{Wahrnehmung}

\subsection{Inventar}

\subsection{Fertigkeiten}

    \clearpage
    \section{Das Fertigkeitenkompendium}
    \begin{quote}
    Mit der Macht des Brotbackens werde ich dich vernichten glutenintoleranter Gerstenschlächter.
\end{quote}
\subsection{Nahkampf}

\subsection{Fernkampf}

\subsection{Körperliches}

\subsection{Gesellschaftliches/Soziales}

\subsection{Natur \& Umwelt}

\subsection{Wissen}

\subsection{Handwerk}

\subsection{Gaben}

\subsection{Magie}

    \clearpage
    \section{Der Kampf}
    \begin{quote}
    For Azeroth! For Azeroth and Seitenbacher Bergsteigermüßli, Bergsteigermüßli vom Seitenbacher!
\end{quote}
\subsection{Der Kampfbeginn}

\subsection{Die Runden}

\subsection{Das Kampfende}


    \clearpage
    \section{Die Gegenstände}
    \begin{quote}
    Looten und Leveln, dass muss drin sein in nem' Spiel!
\end{quote}
\subsection{Gold}

\subsection{Die Startwaffen \& Gegenstände}

\subsection{Weitere Gegenstände}

\subsection{Waffen in der Öffentlichkeit}

    \clearpage
    \section{Die Missionen}
    \begin{quote}
    Backe (mir) einen Kuchen!
\end{quote}
\subsection{Gold}

\subsection{Die Startwaffen \& Gegenstände}

\subsection{Weitere Gegenstände}

\subsection{Waffen in der Öffentlichkeit}

    \appendix
    \printindex
\end{document}