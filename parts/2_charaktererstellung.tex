%
\def\cimp#1{\index{Charakter!Attribute!#1}\imp{#1}}
%
\begin{quote}
    Wir begrüßen dich, Hans, Advokat der ersten Senf-Brigade.
\end{quote}
\subsection{Allgemeines}
\index{Charakter}\index{Charakter!Attribute}Die grundlegenden Eigenschaften eines Charakters ändern sich in diesem Regelwerk nicht wirklich. Überraschung!

Ihr braucht einen \cimp{Namen}, dessen Zusammensetzung irgendwie ins Setting passen sollte. Zudem erfragt der \link[par:charMeta]{Charakterbogen} das \cimp{Alter}, die \cimp{Größe}, das \cimp{Gewicht} und die \cimp{Hautfarbe} eures Charakters. Während die Stimmigkeit der ersten drei (im Bezug auf eure anderen Fertigkeiten) euch überlassen ist, so kann eure Hautfarbe im Bezug zum Setting, rassistisch bedingt, Konsequenzen haben.
Ich empfehle euch, keinen dunkelhäutigen Charakter zu spielen, da dies in der Öffentlichkeit zu Problemen führen kann, möchte euch aber diesbezüglich keinesfalls einschränken.

Weiter sind eure \cimp{Lebenspunkte} von Relevanz.
Ihr alle habt grundsätzlich \(40\) Lebenspunkte.
Diese lassen sich selbstverständlich erhöhen und sie erhöhen sich auch mit eurem \cimp{Level}, wie genau das allerdings im Detail abläuft erfahrt ihr im Kapitel \say{\link[sec:charakterbogen]{Charakterbogen}}.
Elementare Bestandteil eures Wesens ist eure \cimp{Profession}, die vereinfacht das Wesen eures Charakters repräsentiert.
Diese hilft mir bei Fragen zu verstehen was ihr mit eurem Charakter erreichen und welche Rolle ihr einnehmen wollt, ohne euch in die klassische Assassinen-Kampfmagier-Affenzähmer-Virtuosen Klasse einzuteilen. Die Profession ist allerdings auch so relevant, da ich euch darüber und über euer Charakterdesign doch in eine Schublade stecke und dementsprechend Boni verteile. Dies sollte euch helfen, trotz der wenigen Punkte die es zu Beginn gibt, euren Charakter zu verwirklichen. 

Aufgewertet wird euer Charakter, neben einem wunderschönen \cimp{Bild}, durch eure \cimp{Hintergrundgeschichte} (\link[sec:charakterbogen]{Charakterbogen}).
Diese ist in zwei Teile aufgebaut.
Einer der Beiden befindet sich auf eurem Charakterbogen und symbolisiert das von eurem Charakter, was ihr als dieser Preisgebt, bzw. Leute relativ schnell erfahren.
Den anderen, meist durchaus größeren Teil, dürft ihr aufschreiben wie es euch beliebt. 
-- Dieser sollte aber auch mir irgendwie übermittelt werden, damit ich während des Spiels darauf eingehen kann\footnote{Natürlich steht es euch frei den Teil dennoch zu verbreiten, das sollte ich dann nur wissen.}.
Es ist durchaus möglich, dass die Geschichte die ihr auf eurem Charakterbogen verfasst nicht mit dem Extra-Blatt übereinstimmt. Auf dem angegeben Extra-Blatt muss dann aber eure wahre Hintergrundgeschichte stehen, die ihr im Verlauf des PnP auch gerne preisgeben könnt. 
Aber bleibt realistisch, niemand setzt sich so einfach jemandem gegenüber und erzählt von seiner \say{harten Schule einer echten Waise} seitdem er damals aus der Wiege gefallen ist (hust Morgengrau hust \Tongey).
Bitte beachtet, dass es keine gesellschaftliche Stellung gibt, da ihr ja alle in der Gang gelandet seid und diese dementsprechend klar ist. Sollte die Stellung in zukünftigen Sitzungen von Relevanz sein, füge ich sie gegebenenfalls hinzu.


