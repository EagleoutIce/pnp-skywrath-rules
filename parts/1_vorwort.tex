\label{sec:vorwort}Erstmal ja, das hier ist schon ein langer, seeehr langer Text, den ich mit all den Wortwitzen und Versuchen ihn aufzulockern vermutlich nur noch schlimmer und grausiger zu lesen gemacht habe. Aber: Es ist nun mal klausurrelevant, also müssen wir da alle irgendwie durch\ldots 

Diese Vorwort widmet sich hier nur der Illustration beziehungsweise eurer Mitarbeit und einem großen und gütigem Angebot welches ich euch in diesem Hinblick unterbreite:
Überall in diesem Regelwerk habe ich fehlende Grafiken und unfertige Texturen versteckt, die ihr gerne mit euren Fertigkeiten in der Kunst befüllen und veredeln dürft. Wenn ihr also zum Beispiel die hier aufgeführten Zauber (alles unterkringelte ist hierbei nur kurz angemerkt in den meisten PDF-Ansichtsprogrammen anklickbar :D) mit Bildchen versehen wollt, so seid herzlich dazu angestiftet und bereits im vor raus mit Dank überschüttet.
Gleiches gilt allerdings auch für Objekte über das Regelwerk hinaus! Sofern ihr dies wollt (und ich werde mich hier einfach nach einer, für alle verpflichtenden, Mehrheitsentscheidung richten) werde ich die Items die ihr erhaltet -- mit den dazugehörigen Werten -- auf einen Zettel schreiben (bzw. auch das dürft natürlich ihr tun, ganz wie es euch beliebt) und euch die Möglichkeit geben, diese mit illustren Bildern ganz in eurem Stil zu versehen. Dies bedeutet nicht, dass ihr, sobald ihr ein Item in eure gierigen Griffel bekommt, sofort mit dem pinseln anfangen sollt, aber es soll euch gewährt sein in Zeiten der Langeweile oder auch zwischen den Abenteuern das Handwerk der Kunst auszuführen. Das ermöglicht nicht nur eine erweiterte Atmosphäre, sondern auch weniger komplettes abdriften und weniger Nebengespräche und eröffnet zugleich die Möglichkeit dass wir das Abenteuer am Ende verkaufen und damit Reich werden! -- Sorry, abgedriftet, Raphael ist ein schlechter Einfluss \Laughey.

Ich habe auch schon viel zu lange geredet, deswegen seid ihr hiermit in den folgenden Roman entlassen. Lehnt euch zurück, schnappt euch ne Tüte überteuertes Popcorn, Stifte raus und mitgeschrieben, auf-dass ein wundervolles Pen and Paper Abenteuer gedeiht\ldots


Mit der Übersetzung des Regelwerk in \LaTeXe{} gibt es eine Menge Verbesserungen. Darunter einige \link[sec:vorwort]{Querverweise} wie auch einen \link[sec:index]{Index} am Ende des Dokuments.